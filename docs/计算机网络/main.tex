\documentclass[oneside]{book}
\usepackage[fontset=ubuntu,heading=true]{ctex}

\usepackage{graphicx}
\usepackage[most]{tcolorbox}
\usepackage{xcolor} 
\usepackage{tabularray}
\usepackage{amsmath}
\usepackage{amssymb}
\usepackage{caption}
\usepackage[bookmarksnumbered]{hyperref}
\usepackage{enumitem}
\usepackage[
  a4paper,
  left=1cm,
  right=1cm,
  top=2cm,
  bottom=2cm,
]{geometry}

\hypersetup{
    hidelinks,
}

\setlength{\parindent}{2em}

\newcommand{\redbox}[2][]{%
    \begin{tcolorbox}[
        enhanced,              % 启用增强功能
        colback=red!10,        % 浅红色背景
        colframe=red!10!, % 红色边框
        colbacktitle=red!30,   % 标题背景色
        coltitle=black,        % 标题文字颜色
        title=#1,              % 标题内容
        fonttitle=\bfseries,   % 标题加粗
        boxrule=1pt,           % 边框粗细
        arc=4pt,               % 圆角半径
        boxsep=5pt,            % 内边距
        left=6pt, right=6pt,   % 左右边距
        top=4pt, bottom=4pt,   % 上下边距
        breakable,             % 允许跨页
        before skip=10pt,      % 框前间距
        after skip=10pt       % 框后间距
    ]
    #2
    \end{tcolorbox}%
}

\newcommand{\red}[1]{\textcolor{red!80}{#1}}

\newcommand{\redtext}[1]{\textcolor{red!80}{#1}}

\newcommand{\shortcut}[2][]{%
    \begin{tcolorbox}[
        enhanced,              % 启用增强功能
        colback=blue!10,        % 浅红色背景
        colframe=blue!10!, % 红色边框
        colbacktitle=red!30,   % 标题背景色
        coltitle=black,        % 标题文字颜色
        title=#1,              % 标题内容
        fonttitle=\bfseries,   % 标题加粗
        boxrule=1pt,           % 边框粗细
        arc=4pt,               % 圆角半径
        boxsep=5pt,            % 内边距
        left=6pt, right=6pt,   % 左右边距
        top=4pt, bottom=4pt,   % 上下边距
        breakable,             % 允许跨页
        before skip=10pt,      % 框前间距
        after skip=10pt       % 框后间距
    ]
    #2
    \end{tcolorbox}%
}

\let\leq\leqslant
\let\geq\geqslant

\title{计算机网络}
\author{xiaochuan}
\date{\today}

\begin{document}

\frontmatter
\maketitle
\tableofcontents
\mainmatter

\chapter{物理层}
\input{chapter1/chapter1.tex}

\chapter{链路层}
\section{链路层结构}

\begin{enumerate}
    \item 逻辑链路控制层LLC
    \item 介质访问控制层MAC 
\end{enumerate}

\section{逻辑链路层LLC}

服务

\begin{enumerate}
    \item 成帧
    \item 差错控制(可靠数据传输)
    \item 流量控制
\end{enumerate}


\section{成帧}

\begin{tblr}{
colspec={X[1] X[5]},
hlines,
vlines,
}
字节计数 & 通过帧头指出字节数,很少用,出错了很难查出具体的出错位置 \\
标志字节填充 & 通过ESC字符字节加标志分割的字节来界定,原始文件中的ESC前同样需要加ESC字节。PPP协议使用标志字节填充 \\
标志比特填充 & 开始和结束使用一个特殊的比特01111110也就是十六进制7E来分割,内部bit流如果出现5个1就在后面加一个0 \\
物理层编码违禁法 & 在物理层编码时(数据->数字信号符号),可能数字信号的符号数还有剩下的,比如4B/5B编码用5个bit一组的数字信号符号编码一个4bit的数据,还剩下16个数字信号符号没用,可以用来界定帧开始和结束,以太网使用。
\end{tblr}

\section{差错控制}

\subsection{奇偶校验}

校验位长度1,加上校验位中1的个数为奇数为奇校验码,加上校验位中1的个数为偶数为偶校验码

奇校验:所有位数的异或的取反

偶校验:所有位数的异或

只能检出1位的错误,多位的不行,接收方通过计算校验码一样的算法计算出0或者1,0表示无错,1表示有错

\subsection{CRC校验}

使用给定的G生成多项式,用数据除这个生成多项式,使用模2除法,得到的余数是校验码

数据长度为k,校验码长度为r,生成多项式的长度为r+1。实际计算中使用移位和异或

\begin{enumerate}
    \item 数据左移r位
    \item 生成多项式对齐数据的最高位(最高为1的位置)
    \item 异或(此时数据的最高位会变成0)
    \item 重复23步直到数据的有效长度小于等于r
\end{enumerate}

检错能力:所有奇数个错,所有突发错误(从开始到结束的bit位长度小于等于r)

\subsection{海明码}

是一种分组多重偶校验,原数据位为$k$位,校验码长度$r$位,满足不等式 $ k + r + 1 \leq 2^r $ ,这个式子表示校验码能表示的状态数应该大于等于校验后数据改变1位的状态( $ n = k + r $ )加上数据原状态($1$)

\begin{enumerate}
    \item 计算校验码长度
    \item 在原数据中找到每个校验码位所处位置P ,所处位置是2的幂次,从左往右依次是1到k+r,这个P也是偶校验码中所有位置的二进制表示中低P位为1的为一组
    \item 每组进行偶校验获得校验码填入校验位
\end{enumerate}

例:数据 1101

\begin{enumerate}
    \item 根据 $k+r+1 \leq 2^r $有 $4+r+1\leq 2^r$,$r=3$,总共7位,位置1(校验码1) 位置2(校验码2) 位置3(1) 位置4(校验码3) 位置5(1) 位置6(0) 位置7(1)
    \item 校验位分别占位 位置1 位置2 位置4, 
    \item 分组1:位置1(1),位置3(11),位置5(101),位置7(111), 校验码1 = 1\^1\^1 = 1, 分组2:位置2(10),位置3(11),位置6(110),位置7(111),校验码2=1\^0\^1 = 0,分组3,位置4(100),位置5(101),位置6(110),位置7(111),校验码3= 1 \^ 0 \^ 1 = 0
\end{enumerate}

最后的海明码为1010101

接收放校验的时候也是一样的逻辑,然后对每组进行偶校验

标准海明码纠错1位检错2位

\section{流量控制}

停止等待协议和滑动窗口协议

\section{可靠数据传输}

\begin{tblr}{
colspec={X[1] X[1] X[1] X[5]},
hlines,
vlines,
}
协议 & 发送窗口大小 & 接受窗口大小 & 详细 \\
停止等待协议 & 1 & 1 & 超时计时器,ack,帧编号 \\
回退N帧协议 & n & 1 & 使用累计确认,有错重传上一个ack之后的帧片段 \\
选择重传 & n & n & 使用nak,只重传出错的帧
\end{tblr}
\vspace{8pt}
传输率 = 发送时间/一个发送周期的时间
\vspace{8pt}
\begin{tblr}[caption="传输率"]{
colspec={X[1] X[1] X[1] X[5]},
hlines,
vlines,
}
停止等待协议 &  发送时间/发送时间+接收时间+RTT \\
回退N帧协议 &  n * 发送时间/(发送时间+接收时间+RTT, n是发送窗口大小) \\
选择重传 & 1
\end{tblr}

\section{介质访问控制子层MAC}

在广播信道出现,多个设备共享同一条链路

\subsection{静态介质访问控制}

通过物理层的复用实现(见上节)

\subsection{动态介质访问控制}

\subsubsection{ALOHA}

\begin{tblr}{
colspec={X[1] X[5]},
hlines,
vlines,
}
ALOHA &  原始的动态介质访问控制,随机发送,有冲突后随机再发 \\
时隙ALOHA &  信道分为等长时隙,只有在时隙开始时才能发送,发送一帧的时间小于等于时隙
\end{tblr}

\subsubsection{CSMA}
\begin{tblr}{
colspec={X[1] X[5]},
hlines,
vlines,
}
1坚持CSMA &  信道空闲发送,信道忙则侦听至信道空闲 \\
非坚持CSMA &  信道空闲发送,信道忙则随机等待一段时间再开始监听 \\
p-坚持CSMA & 适用于时隙信道,信道空闲有p概率发送,1-p概率到下一个时隙监听,信道忙则下一个时隙继续监听,
\end{tblr}

\subsubsection{CSMA/CD}

带冲突检测的CSMD,通过电压变化来检测冲突,发送时持续侦听。检测到碰撞后使用截断二进制指数退避算法。至少要2倍传播时延才能确定有没有冲突,争用期就是最小能检测到冲突的时间,也就是必须持续侦听2倍传播时延,即
$$
\text{最小帧长} = \text{把数据从网卡发送到链路的速度(数据传输率)} \times \text{2倍传播时延}
$$

截断二进制指数退避算法

\begin{enumerate}
    \item 取争用期作为基本时间
    \item 当前已重传次数为$n$, $k = \min(10, n+1)$, $r = rand(0, 2^{k-1})$
    \item 取$r$倍的基本时间为推迟时间
\end{enumerate}

\subsubsection{CSMA/CA}

碰撞避免。使用虚拟载波侦听,使用RTS(Request To Send),CTS(Clear To Send)来预约信道。发送方发送RTS,接收方广播CTS

\section{以太网}

\begin{center}
\includegraphics[width=0.7\textwidth]{pics/p2.1.png}
\end{center}

只有以太网的帧地址目的地址在前源地址在后

\section{802.11无线网}

使用正交频分复用(OFDM)

\begin{enumerate}
    \item 分布式模型
    \item 自组织模型(用的很少)
\end{enumerate}

\subsection{帧间间隔}

在监听到信道空闲后还需要等待一段时间才能发送帧,叫帧间间隔

\begin{tblr}{
colspec={X[1] X[5]},
hlines,
vlines,
}
SIFS (Shortest InterFrame Space) & 优先级最高,CTS,ACK帧使用  \\
PIFS &  在PCF模式下工作,介于DIFS和SIFS \\
DIFS & 正常数据帧和RTS帧使用
\end{tblr}

\subsection{帧结构}

\begin{center}
\includegraphics[width=0.6\textwidth]{pics/p2.2.png}
\end{center}

DS是AP背后的有线网络

\begin{tblr}{
colspec={X[1] X[1] X[5]},
hlines,
vlines,
}
Address 1 & 接收者地址 & 总是表示这一帧在当前传输阶段的直接接收者(即下一个要接收此无线电波的设备)。  \\
Address 2 & 发送者地址 & 总是表示这一帧在当前传输阶段的直接发送者(即发出此无线电波的设备)。 \\
Address 3 & 过滤/路由地址 & 这是最核心的地址。通常用于AP在无线和有线网络之间进行地址过滤或路由。在大多数情况下,它承载着最终目的地址或原始源地址。\\
Address 4 & 特殊情况下使用 & 仅在无线分布式系统中,数据在两个AP之间桥接时使用。
\end{tblr}


\begin{tblr}{
  colspec = {X[2] X[1] X[1] X[4] X[1] X[1] X[1] X[1]},
  hlines,
  vlines,
}
场景 & To DS & From DS & 场景描述 & Address 1 (接收者) & Address 2 (发送者) & Address 3 (过滤/路由) & Address 4 \\
1. 站到站 & 0 & 0 & 同一个Wi-Fi网络内的两个客户端直接通信(Ad-hoc模式,或支持客户端直连)。 & 目的MAC & 源MAC & BSSID & 未使用 \\
2. 客户端发送给AP & 1 & 0 & 客户端(如手机)发送数据给AP,目的是让AP将其转发到有线网络或另一个客户端。 & AP的MAC & 客户端MAC & 目的MAC & 未使用 \\
3. AP发送给客户端 & 0 & 1 & AP从有线网络收到数据,将其发送给目标客户端。 & 客户端MAC & AP的MAC & 源MAC & 未使用 \\
4. 无线中继(WDS) & 1 & 1 & 数据在两个AP之间进行无线桥接(不常见)。 & 接收AP的MAC & 发送AP的MAC & 目的MAC & 源MAC \\
\end{tblr}
注:BSSID(基本服务集标识符)通常就是 AP 的无线接口 MAC 地址。

\section{交换机}

交换机通过\redtext{源MAC地址}来学习填充交换表,如果目的地址不在交换表里面则广播除了发送地址接口的所有接口

\end{document}
